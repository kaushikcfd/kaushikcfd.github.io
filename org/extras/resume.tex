\documentclass[letterpaper, 13pt]{article}

\usepackage{geometry}
\usepackage[bookmarksnumbered,hidelinks]{hyperref}
\usepackage{soul}
\usepackage[T1]{fontenc}
\usepackage{lipsum}
\usepackage{tabularx}
\usepackage[style=ieee, defernumbers=true]{biblatex}
\usepackage{enumitem}
\pagestyle{empty}
\newcommand{\updatenote}[1][\today]{\par\vfill{\scriptsize \textit{Last updated on #1}}}
\addbibresource{journal.bib} 
\addbibresource{conference.bib}

\defbibfilter{papers}{
  type=article or
  type=preprint  
}


% -------------------------------------------------------------
% TODO:
% [ ] Maintain a org-file instead of tex
% [x] Proof read the research paragraphs
% [x] Adding code links to the description
% [x] Change the first section.
% [x] Change the spacing of the empty line in tables
% [x] Reduce the spacing between the advisor and title and ~~increase between the description on advisor.~~
% [x] Change the way you write about research.
% [x] Remove enumerates.
% [x] Add the current research.
% [x] Complete the Morgan Stanley discussion section
% ---------------------------------------------------------------



\begin{document}

% {{{ Title

\begin{center}
    \LARGE
    \textsc{\textbf{Kaushik G Kulkarni}}\\
    \vspace{1ex}
    \normalsize
    \begin{tabular}{p{0.1\textwidth}p{8ex} p{0.4\textwidth}}
        & Address:    & 201N Goodwin Ave.,\\
        &             & Urbana, IL 61801,\\
        &             & USA\\
        & Email:      &\href{mailto:kgk2@illinois.edu}{kgk2@illinois.edu}\\
        & Web:        & \texttt{\small \url{https://kaushikcfd.github.io}} \\
    \end{tabular}
\end{center}

% }}}

% {{{ Education

\section*{\Large\textsc{Education}}
\vspace{-4ex}
\rule{\textwidth}{0.1ex}\\
\vspace{1ex}\\
\begin{tabular}{p{0.25\textwidth} p{0.67\textwidth}}
Aug '17 --  \textit{Present}  & Ph.D. student in Computer Science \\
                              & \textit{University of Illinois at Urbana-Champaign, Urbana, IL} \\
                              & \textit{Adviser: Prof. Andreas Kl\"{o}ckner}\vspace{1ex}\\
Aug '13 -- May '17            & Bachelor of Technology in Mechanical Engineering \\
                              & \textit{Indian Institute of Technology, Bombay} \\
                              & \textit{Adviser: Prof. Shiva Gopalakrishnan}
\end{tabular}

% }}}

% {{{ Work Experience

\section*{\Large\textsc{Experience}}
\vspace{-4ex}
\rule{\textwidth}{0.1ex}\\
\vspace{1ex}\\
\begin{tabular}{p{0.25\textwidth} p{0.69\textwidth}}
Aug '20 -- \textit{Present} &   Research Assistant\\
                            &   \textit{\small National Center for
                            Supercomputing Applications, UIUC}\vspace{1ex}\\
Aug '17 -- Aug' 20          &   Graduate Research/Teaching Assistant\\
                            &   \textit{\small Computer Science Dept., UIUC}\vspace{1ex}\\
May '19 -- Aug '19          &   Givens Associate\\
                            &   \textit{\small Argonne National Laboratory, Illinois}\\
                            & {\small Improved the implementation of Nonlinear optimization algorithms on multi-core systems.}\vspace{1ex}\\
Jan '16 -- May '17          &   Teaching Assistant for Introduction to Numerical Analysis\\
                            &   \textit{\small Mathematics Dept., IITB}\vspace{1ex}\\
May '16 -- July '16         &   Software Engineering Intern\\
                            &   \textit{\small Morgan Stanley Strats and Modelling, Mumbai}\\
                            & Developed internal tools to aid cash flow visualizations for traders.
\end{tabular}

% }}}

% {{{ Publications and conferences


\nocite{*}
\printbibliography[title={\Large\textsc{Publications}\vspace*{-2ex}\\\rule{\textwidth}{0.1ex}},filter=papers]
\printbibliography[title={\Large\textsc{Talks}\vspace*{-2ex}\\\rule{\textwidth}{0.1ex}},type=inproceedings, resetnumbers=true]

% }}}

% {{{ Awards and Achievements

\section*{\Large\textsc{Awards and Achievements}}
\vspace{-4ex}
\rule{\textwidth}{0.1ex}
\vspace{1ex}\\
\begin{tabular}{p{0.25\textwidth} p{0.67\textwidth}}
  2022    &   Graduate College Presentation Award to present at SIAM Conference
    on Parallel Processing for Scientific Computing \vspace{1ex}\\
  2021    &   SIAM Travel Award for Conference on Computational Science and
    Engineering \vspace{1ex}\\
  2019    &   SIAM Travel Award for Conference on Computational Science and
    Engineering (Spokane, WA) \vspace{1ex}\\
  2016    &   Undergraduate Research Award (URA01) by Indian Institute of
    Technology, Bombay \vspace{1ex}\\
  2013    &   \textit{Kishore Vaigyanik Protsahan Yojana} Fellowship
    Award\vspace{1ex} \\
  2013    &   Certificate of Merit for being among the State Top 1\% in National
    Standard examination in Physics and Chemistry \\
\end{tabular} 

% }}}

% {{{ Research

\section*{\Large\textsc{Research}}

\vspace{-4ex}
\rule{\textwidth}{0.1ex}

\subsection*{Optimizing Einstein-Summation Subprograms}
\vspace{-1ex}
\small \textit{Advised by Prof.~Andreas Kl\"{o}ckner, UIUC}\\
Einstein-summations provide a simple notation to express a wide-range of
Linear-Algebra primitives. Achieving roofline FlOp-throughput for these
operations still remains challenging. We employ a combination of techniques
from auto-tuning to pattern-matching to generate efficient \texttt{OpenCL} code for
computational kernels containing expressions that have \texttt{einsum}-like memory
access patterns.

\subsection*{Array Programming Languages and Intermediate Representations}
\vspace{-1ex}
\small \textit{Advised by Prof.~Andreas Kl\"{o}ckner, UIUC}\\
Array Programming Languages have been an important vehicle for driving
scientific applications from as early as the 1960s. Besides
providing a close-to-math expressibility, their intermediate representations
are closer to SIMD architectures making it easier to engineer optimizing
compilers targeting such hardwares.
\textsc{Pytato} provides one such IR that
lowers $n-d$ array programs to computation graphs comprising of \textit{pure}-Array
Ops that can be targeted to \texttt{OpenCL} / \texttt{CUDA} / \texttt{JAX}.

\subsection*{Near-Roofline Discontinuous Galerkin Action Operators}
\vspace{-1ex}
\small \textit{Advised by Prof.~Andreas Kl\"{o}ckner, UIUC}\\
Discontinuous Galerkin operator applications comprise of many fine-grained
array operations that can push them into the memory-bound regime. With kernel
and loop fusion we can bump up the workload's Arithmetic Intensity, however,
performing fusion might also negatively affect the kernel's performance by
inhibiting device's latency hiding abilities by further introducing dependency
edges and increasing the working set size of the inner loops. In the \textsc{MIRGE-Com}
framework we address such trade-offs for GPU systems.

\subsection*{Finite Element Assembly on GPUs}
\vspace{-1ex}
\small \textit{Advised by Prof.~Andreas Kl\"{o}ckner, UIUC}\\
Evaluation of Finite Element operators result in a diverse set of
computational kernels making it a difficult problem to find one optimization
strategy that achieves near-peak performance for all the kernels on GPUs.  We
solve this problem by using high level code generation tools that select the
optimization strategy based on the loop structure of the kernel.


\subsection*{Solving Eikonal Equations on Unstructured Grids}
\vspace{-1ex}
\small \textit{Advised by Prof.~S Baskar, IIT Bombay}\\
Characteristic Fast Marching Method is widely used in solving the Eikonal
equations, however previous work had been only formulated for structured
grids. We developed a solver that extended the algorithm for
unstructured grids as well. Used the solver to solve known problems in
literature with skew grids so that the activity of the solution could be
efficiently observed in the region of activity.\\
Link: \texttt{https://github.com/kaushikcfd/eikonal-unstructured}


\subsection*{Discontinuous Galerkin Framework for Hyperbolic PDEs}
\vspace{-1ex}
\small \textit{Advised by Prof.~Shiva Gopalakrishnan, IIT Bombay}\\
We developed a \texttt{C++} library for solving Hyperbolic Equations through
Discontinuous Galerkin (``DG'') methods on structured grids.  Performed a
series of convergence tests to verify that the framework satisfied
$hp-$convergence.
Eventually, used the framework to simulate problems in Fluid Dynamics like the
dam-break problem using high order DG elements.\\
Link: \texttt{https://github.com/kaushikcfd/Discontinuous-Galerkin}


\subsection*{Flow Induced Reconfiguration of Aquatic Vegetation}
\vspace{-1ex}
\small \textit{Advised by Prof.~Rajneesh Bharadwaj, IIT Bombay}\\
Corrected the existing models for Fluid Structure Interaction for a Flexible
plate by including the Skin friction coefficient in the computations.
Implemented a ``\textit{Predictor-Corrector}'' based Finite Difference scheme
for the computation of coefficient of drag on the plate.\\
Link: \texttt{https://arxiv.org/abs/1712.00441}

% }}}

\updatenote{}
\end{document}

% vim:foldmethod=marker
