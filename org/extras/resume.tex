\documentclass[letterpaper, 12pt]{article}
\usepackage{geometry}
\usepackage[bookmarksnumbered,hidelinks]{hyperref}
\usepackage{soul}
\usepackage[T1]{fontenc}
\usepackage{lipsum}
\usepackage{tabularx}
\usepackage[style=ieee]{biblatex}
\usepackage{enumitem}
\pagestyle{empty}
\newcommand{\updatenote}[1][\today]{\par\vfill{\scriptsize \textit{Last updated on #1}}}
\addbibresource{publications.bib}

% -------------------------------------------------------------
% TODO:
% [ ] Maintain a org-file instead of tex
% [x] Proof read the research paragraphs
% [x] Adding code links to the description
% [x] Change the first section.
% [x] Change the spacing of the empty line in tables
% [x] Reduce the spacing between the advisor and title and ~~increase between the description on advisor.~~
% [x] Change the way you write about research.
% [x] Remove enumerates.
% [x] Add the current research.
% [x] Complete the Morgan Stanley discussion section
% ---------------------------------------------------------------



\begin{document}

% {{{ Title

\begin{center}
    \LARGE
    \textsc{\textbf{
    Kaushik G Kulkarni}}\\
    \vspace{1ex}
    \normalsize
    \begin{tabular}{p{0.1\textwidth}p{8ex} p{0.4\textwidth}}
        & Address:    & 201N Goodwin Ave.,\\
        &             & Urbana, IL 61801,\\
        &             & USA\\
        & Email:      &\href{mailto:kgk2@illinois.edu}{\texttt{kgk2@illinois.edu}}\\
        & Web:        & \texttt{\small \url{https://kaushikcfd.github.io}} \\
        & Phone:      &\texttt{(+1)-217-6488653}
    \end{tabular}
\end{center}

% }}}

% {{{ Education

\section*{\Large\textsc{Education}}
\vspace{-4ex}
\rule{\textwidth}{0.1ex}\\
\vspace{1ex}\\
\begin{tabular}{p{0.25\textwidth} p{0.67\textwidth}}
Aug '17 --  \textit{Present}  & Ph.D. student in Computer Science \\
                              & \textit{University of Illinois at Urbana-Champaign, Urbana, IL} \\
                              & \textit{Adviser: Prof. Andreas Kl\"{o}ckner} \\
                              & \\
Aug '13 -- May '17            & Bachelor of Technology in Mechanical Engineering \\
                              & \textit{Indian Institute of Technology, Bombay} \\
                              & \textit{Adviser: Prof. Shiva Gopalakrishnan}\\
\end{tabular}

% }}}

% {{{ Work Experience

\section*{\Large\textsc{Experience}}
\vspace{-4ex}
\rule{\textwidth}{0.1ex}\\
\vspace{1ex}\\
\begin{tabular}{p{0.25\textwidth} p{0.69\textwidth}}
Dec '18 -- \textit{Present} &   Graduate Research Assistant \\
                            &   \textit{\small Computer Science Dept., UIUC}\vspace{1ex}\\
Aug '18 -- Dec '18          &   Teaching Assistant for Numerical Analysis\\
                            &   \textit{\small Computer Science Dept., UIUC}\vspace{1ex}\\ 
Dec '17 -- Aug '18          &   Graduate Research Assistant \\
                            &   \textit{\small Computer Science Dept., UIUC}\vspace{1ex}\\
Aug '17 -- Dec '17          &   Teaching Assistant for Numerical Methods\\
                            &   \textit{\small Computer Science Dept., UIUC}\vspace{1ex}\\
Jan '17 -- May '17          &   Teaching Assistant for Introduction to Numerical Analysis\\
                            &   \textit{\small Mathematics Dept., IITB}\vspace{1ex}\\
May '16 -- July '16         &   Software Engineering Intern\\
                            &   \textit{\small Morgan Stanley Strats and Modelling, Mumbai}\\
\end{tabular}

% }}}

% {{{ Publications

\nocite{*}
\printbibliography[title={\Large\textsc{Publications}\vspace*{-2ex}\\\rule{\textwidth}{0.1ex}}]

% }}}

% {{{ Awards and Achievements

\section*{\Large\textsc{Awards and Achievements}}
\vspace{-4ex}
\rule{\textwidth}{0.1ex}
\vspace{1ex}\\
\begin{tabular}{p{0.25\textwidth} p{0.67\textwidth}}
2019    &   Travel Award for SIAM Conference on Computational Science and Engineering(Spokane, WA) \vspace{1ex}\\
2016    &   Undergraduate Research Award(URA01) by Indian Institute of Technology, Bombay \vspace{1ex}\\
2013    &   All India Rank of \textit{341} (top 0.01\%) in the Joint Entrance Examination - Mains out of 1.5 million students \vspace{1ex} \\
2013    &   All India Rank of \textit{419} (top 0.3\%)in the Joint Entrance Examination - Advanced out of 0.15 million students\vspace{1ex} \\
2013    &   \textit{Kishore Vaigyanik Protsahan Yojana} Fellowship Award\vspace{1ex} \\
2013    &   Certificate of Merit for being among the State Top 1\% in National Standard examination in Physics and Chemistry \\
\end{tabular} 

% }}}

% {{{ Research

\section*{\Large\textsc{Research}}
\vspace{-4ex}
\rule{\textwidth}{0.1ex}


\subsection*{Finite Element Assembly on GPUs}
\vspace{-1ex}
\small \textit{Advised by Prof. Andreas Kl\"{o}ckner, UIUC}\\
\textit{Describe the problem and state what you did.}
Finite Element Assembly ...


\subsection*{Abstractions for High Performance Computing}
\vspace{-1ex}
\small \textit{Advised by Prof. Andreas Kl\"{o}ckner, UIUC}\\
\textit{Describe the problem and state what you did.}


\subsection*{Solving Eikonal Equations on Unstructured Grids}
\vspace{-1ex}
\small \textit{Advised by Prof. S Baskar, IIT Bombay}\\
Characteristic Fast Marching Method is widely used in solving the Eikonal
equations, however previous work had been only formulated for structured
grids. We developed a solver that extended the algorithm for
unstructured grids as well. Used the solver to solve known problems in
literature with skew grids so that the activity of the solution could be
efficiently observed in the region of activity.\\
Link: \texttt{https://github.com/kaushikcfd/eikonal-unstructured}


\subsection*{Discontinuous Galerkin Framework for Hyperbolic PDEs}
\vspace{-1ex}
\small \textit{Advised by Prof. Shiva Gopalakrishnan, IIT Bombay}\\
We developed a \texttt{C++} library for solving Hyperbolic Equations through
Discontinuous Galerkin (``DG'') methods on structured grids.  Performed a series
of convergence tests to verify that the framework satisfied $hp-$convergence.
Eventually, used the framework to simulate problems in Fluid Dynamics like
the dam-break problem using high order DG elements.\\
Link: \texttt{https://github.com/kaushikcfd/Discontinuous-Galerkin}


\subsection*{Flow Induced Reconfiguration of Aquatic Vegetation}
\vspace{-1ex}
\small \textit{Advised by Prof. Rajneesh Bharadwaj, IIT Bombay}\\
Corrected the existing models for Fluid Structure Interaction for a Flexible plate by including the Skin friction coefficient in the computations. Implemented a "\textit{Predictor-Corrector}" based Finite Difference scheme for the computation of coefficient of drag on the plate.\\
Link: \texttt{https://arxiv.org/abs/1712.00441}

% }}}

\updatenote
\end{document}

% vim:foldmethod=marker
